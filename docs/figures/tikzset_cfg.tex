\usepackage{pgf}

% Debug to print label to nodes
\newif\ifdebug
\debugtrue
\debugfalse

%%%% CONFIG
\def\debugFontSize{\tiny}
%%%%


%% Debug node
% #1: test
% #2: position
\newcommand{\debugN}[2][]{
    \node at #2 {\ifdebug o \else \fi};
    \node [above] at #2 {\debugFontSize \ifdebug #1 \else \fi};
}

% Basic rectangle:
% #1: draw param
% #2: uid
% #3: center
% #4: width 
% #5: height
% #6: amount of nodes on W side
% #7: amount of nodes on E side
% #8: amount of nodes on N side
% #9: amount of nodes on S side
\newcommand{\rectangle}[9][]{
    \coordinate (#2/TL) at ($#3 + (-#4/2,#5/2)$);
    \coordinate (#2/TR) at ($#3 + (#4/2,#5/2)$);
    \coordinate (#2/BL) at ($#3 + (-#4/2,-#5/2)$);
    \coordinate (#2/BR) at ($#3 + (#4/2,-#5/2)$);
    \draw [#1] (#2/TL) -- (#2/TR)%
    -- (#2/BR) -- (#2/BL) -- (#2/TL); 
    % Draw the nodes on side W
    \foreach \xi in {1,...,#6}{
        \pgfmathsetmacro\offy{(#5)/(#6+1)}
        \pgfmathsetmacro\yval{\xi*\offy}
        \coordinate (#2/W\xi) at ($(#2/TL) - (0,\yval)$);
        \debugN[W\xi]{(#2/W\xi)}
    }
    % Draw the nodes on side E
    \foreach \xi in {1,...,#7}{
        \pgfmathsetmacro\offy{(#5)/(#7+1)}
        \pgfmathsetmacro\yval{\xi*\offy}
        \coordinate (#2/E\xi) at ($(#2/TR) - (0,\yval)$);
        \debugN[E\xi]{(#2/E\xi)}
    }
    % Draw the nodes on side N
    \foreach \xi in {1,...,#8}{
        \pgfmathsetmacro\offx{(#4)/(#8+1)}
        \pgfmathsetmacro\xval{\xi*\offx}
        \coordinate (#2/N\xi) at ($(#2/TL) + (\xval,0)$);
        \debugN[N\xi]{(#2/N\xi)}
    }
    % Draw the nodes on side S
    \foreach \xi in {1,...,#9}{
        \pgfmathsetmacro\offx{(#4)/(#9+1)}
        \pgfmathsetmacro\xval{\xi*\offx}
        \coordinate (#2/S\xi) at ($(#2/BL) + (\xval,0)$);
        \debugN[S\xi]{(#2/S\xi)}
    }
    %%%
    \coordinate (#2/center) at #3;
    \coordinate (#2/south) at ($#3 - (0,#5/2)$);
    \coordinate (#2/north) at ($#3 + (0,#5/2)$);
}

%% Rectangle from corner locations
% #1: draw param
% #2: uid
% #3: TR pos
% #4: BL pos
% #5: amount of nodes on W side
% #6: amount of nodes on E side
% #7: amount of nodes on N side
% #8: amount of nodes on S side
\newcommand{\rectangleC}[8][]{
    % Extract width/height from the corners
    \coordinate (sizerC) at ($#3-#4$); 
    % Define coordinate for remaining cornes
    \path let \p1=(sizerC) in coordinate (TL) at ($#4+(0,\y1)$);
    \path let \p1=(sizerC) in coordinate (BR) at ($#4+(\x1,0)$);
    % Draw the rectange
    \draw [#1] #3 |- #4 #3 -| #4;
    % Draw the node on side W
    \foreach \xi in {1,...,#5}{
        \pgfmathsetmacro\propL{\xi/(#5+1)}
        \coordinate (#2/W\xi) at ($(TL)!\propL!#4$);
        \debugN[W\xi]{(#2/W\xi)}
    }
    % Draw the node on side E
    \foreach \xi in {1,...,#6}{
        \pgfmathsetmacro\propL{\xi/(#6+1)}
        \coordinate (#2/E\xi) at ($#3!\propL!(BR)$);
        \debugN[E\xi]{(#2/E\xi)}
    }
    % Draw the node on side N
    \foreach \xi in {1,...,#7}{
        \pgfmathsetmacro\propL{\xi/(#7+1)}
        \coordinate (#2/N\xi) at ($(TL)!\propL!#3$);
        \debugN[N\xi]{(#2/N\xi)}
    }
    % Draw the node on side S
    \foreach \xi in {1,...,#8}{
        \pgfmathsetmacro\propL{\xi/(#8+1)}
        \coordinate (#2/S\xi) at ($#4!\propL!(BR)$);
        \debugN[S\xi]{(#2/S\xi)}
    }
    % Generate general coordinate
    \coordinate (#2/TL) at (TL);
    \coordinate (#2/TR) at #3;
    \coordinate (#2/BL) at #4;
    \coordinate (#2/BR) at (BR);
    \coordinate (#2/center) at ($#3!0.5!#4$);
    \coordinate (#2/south) at ($(#2/BL)!0.5!(#2/BR)$);
}

% Flip-Flop
% 1: draw style
% 2: id 
% 3: loc
% 4: width
% 5: height
\newcommand{\DFF}[5][]{
    % Draw a rectangle
    \rectangle[#1]{rect}{#3}{#4}{#5}{1}{1}{1}{1}
    % Draw the triangle
    \pgfmathsetmacro\baseT{(#4)/3}
    \pgfmathsetmacro\baseTd{\baseT/2}
    \pgfmathsetmacro\heightT{(#5)/7}
    \coordinate (TC0) at ($(rect/south) - (\baseTd,0)$);
    \coordinate (TC1) at ($(rect/south) + (\baseTd,0)$);
    \coordinate (TC2) at ($(rect/south) + (0,\heightT)$);
    \draw [#1] (TC0) -- (TC2) -- (TC1);
    \coordinate (#2/D) at (rect/W1);
    \coordinate (#2/Q) at (rect/E1);
    \coordinate (#2/center) at #3;
    \coordinate (#2/north) at (rect/N1);
}

% XOR
% 1: draw style
% 2: id
% 3: loc
% 4: radius
\newcommand{\XOR}[4][]{
    % Draw a circle    
    \draw [#1] #3 circle (#4);
    % Generate coordinate for th xor
    \coordinate (#2/north) at ($#3 + (0,#4)$);
    \coordinate (#2/south) at ($#3 + (0,-#4)$);
    \coordinate (#2/east) at ($#3 + (#4,0)$);
    \coordinate (#2/west) at ($#3 + (-#4,0)$);
    \coordinate (#2/center) at ($#3$);
    \draw [#1] (#2/north) -- (#2/south);
    \draw [#1] (#2/west) -- (#2/east);
}


%
