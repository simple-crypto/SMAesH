% define a mux
\tikzset{mux2/.style={muxdemux,muxdemux def={Lh=4, NL=2, Rh=3,NB=1,w=1}}}


%% CONFIG
% Size of DFF instance
\def\widthDFF{1.5}
\def\heightDFF{3}
% Spacing between DFF instance
\def\spacexDFF{9.5}
\def\spaceyDFF{12}
% Spacing after DFF and XOr instance
\def\spaceXOR{0}

% Line width of DFF
\def\lwModule{0.7mm}
\def\lwWire{0.5mm}
\def\scaleCTIKZ{0.4}


\def\spacexFeedBack{0.75}
\def\spacexFeedBackR{2}
\def\spaceyXorOut{0.75*\heightDFF}
\def\spaceyXorOutMC{0.45*\heightDFF}
\def\spacexXorOut{2}
\def\spaceyFeedBack{3}

\def\spacexIn{0.5}

\def\fontS{\Large}
\def\fontCtrl{}

\def\xshiftMuxFromSBInv{1cm}

%\debugtrue;

%% Macro for the ctrl signals of the mux2
% 1: mux_id
% 2: control sig
% 3: top value
% 4: bottom value
\newcommand{\muxCtrl}[4]{
    \node[anchor=north,rotate=0] at (#1.bpin 1) {\fontCtrl #2};
    \node at (#1.center up) {\fontCtrl #3};
    \node at (#1.center down) {\fontCtrl #4};
}

% W internal port macro with annotation
% #1 style
% #2 id
% #3 loc
% #4 text
\def\yshw{2mm}
\def\xshw{2mm}
\newcommand{\portW}[4][]{
    \coordinate (#2) at (#3);
    \node [#1,anchor=east,yshift=\yshw] at (#3) {#4};
}
% E internal port macro with annotation
% #1 style
% #2 id
% #3 loc
% #4 text
\newcommand{\portE}[4][]{
    \coordinate (#2) at (#3);
    \node [#1,anchor=west,yshift=\yshw] at (#3) {#4};
}

% Macro for a bloc with Register with mux input
% #1: style
% #2: id
% #3: loc (center DFF)
\newcommand{\DFFMUX}[3][]{
    % draw DFF
    \DFF[line width=\lwModule]{dffinst}{#3}{\widthDFF}{\heightDFF}
    % draw mux
    \node[line width=\scaleCTIKZ*\lwModule,mux2,anchor=rpin 1, xshift=-0.5cm] (#2/mux) at (dffinst/D) {}; 
    % Connector
    \draw[line width=\lwWire] (#2/mux.rpin 1) -- (dffinst/D);
    % draw small port at Q
    \coordinate (#2/out) at ($(dffinst/Q)+(1,0)$);
    \draw [line width=\lwWire] (dffinst/Q) -- (#2/out);
    % Generate remaining instance coordinate
    \coordinate (#2/text) at (dffinst/center);
    \coordinate (#2/in1) at (#2/mux.lpin 1);
    \coordinate (#2/in2) at (#2/mux.lpin 2);
    \coordinate (#2/ctrl) at (#2/mux.bpin 1);
    % Debug node
    \debugN[out]{(#2/out)}
    \debugN[in1]{(#2/in1)}
    \debugN[in2]{(#2/in2)}
    \debugN[text]{(#2/text)}
    \debugN[ctrl]{(#2/ctrl)}
}

% Macro for a bloc with muxex input key, xor and output port
% #1: style
% #2: id
% #3: loc (west of xor)
\newcommand{\KEYXOR}[3][]{
    \def\Radius{0.3}
    % Draw small port at the input of xor
    \coordinate (#2/inxor) at #3;
    \XOR[line width=\lwModule]{xor}{($(#2/inxor)+(\Radius,0)$)}{\Radius}
    \draw [line width=\lwWire] (#2/inxor) -- (xor/west);
    % Draw the XOR node
    \node[line width=\scaleCTIKZ*\lwModule,mux2,anchor=rpin 1] (#2/mux) at ($(xor/south)+(-0.5*\widthDFF,-2-\spaceyFeedBack)$) {};
    \draw [->] [line width=\lwWire] (#2/mux.rpin 1) -| (xor/south);
    % Draw small port at output1
    \coordinate (#2/out1) at (xor/north);
    % Draw small port at output2
    \coordinate (#2/out2) at ($(xor/east) + (0.5,0)$);
    \draw [line width=\lwWire] (xor/east) -- (#2/out2);
    % Generate remaining coordinate
    \coordinate (#2/in1) at (#2/mux.lpin 1);
    \coordinate (#2/in2) at (#2/mux.lpin 2);
    \coordinate (#2/ctrl) at (#2/mux.bpin 1);
    % Debug node
    \debugN[in1]{(#2/in1)}
    \debugN[in2]{(#2/in2)}
    \debugN[out1]{(#2/out1)}
    \debugN[out2]{(#2/out2)}
    \debugN[inxor]{(#2/inxor)}
    \debugN[ctrl]{(#2/ctrl)}
}

%% Macro for a bloc with two input muxes (choosing between from SB and from MC) 
% 1: style
% 2: id 
% 3: loc (out of last mux)
\newcommand{\MUXIN}[3][]{
    % Draw last mux
    \node[line width=\scaleCTIKZ*\lwModule,mux2,anchor=rpin 1] (#2/muxFB) at #3 {};
    \node[line width=\scaleCTIKZ*\lwModule,mux2,anchor=rpin 1,xshift=-1cm] (#2/muxIn) at (#2/muxFB.lpin 1) {};
    \draw [line width=\lwWire] (#2/muxIn.rpin 1) -- (#2/muxFB.lpin 1);
    % Generate coordinate
    \coordinate (#2/out) at #3;
    \coordinate (#2/inFB) at (#2/muxFB.lpin 2);
    \coordinate (#2/inSB) at (#2/muxIn.lpin 1);
    \coordinate (#2/inMC) at (#2/muxIn.lpin 2);
    \coordinate (#2/ctrlFB) at (#2/muxFB.bpin 1);
    \coordinate (#2/ctrlIn) at (#2/muxIn.bpin 1);
    % debug node
    \debugN[out]{(#2/out)}
    \debugN[inFB]{(#2/inFB)}
    \debugN[inSB]{(#2/inSB)}
    \debugN[inMC]{(#2/inMC)}
    \debugN[ctrlFB]{(#2/ctrlFB)}
    \debugN[ctrlIn]{(#2/ctrlIn)}
}

%% Macro for the MC bloc RAW 
% 1: style
% 2: id 
% 3: loc
\newcommand{\MCBlocRAW}[3][]{
    \def\scaleIO{0.7}
    % Draw the MC
    \rectangle[line width=\lwModule]{rec}{#3}{3}{5}{4}{4}{0}{0};
    \coordinate (#2/center) at (rec/center);
    % Ports declaration
    \foreach \xi in {1,...,4}{
        \pgfmathsetmacro\idxLabel{int(\xi-1)}
        % Input 
        \portW{}{rec/W\xi}{\fontS $\dpStateByteToMC\idxLabel$}
        % Output
        \portE{}{rec/E\xi}{\fontS $\dpStateByteFromMC\idxLabel$}
        % Generate node
        \coordinate (#2/a\xi) at ($(rec/W\xi)+(-\scaleIO,0)$);
        \draw [->, line width=\lwWire] (#2/a\xi) -- (rec/W\xi);
        \coordinate (#2/b\xi) at ($(rec/E\xi)+(\scaleIO,0)$);
        \draw [line width=\lwWire] (rec/E\xi) -- (#2/b\xi);
        % debug node
        \debugN[a\xi]{(#2/a\xi)}
        \debugN[b\xi]{(#2/b\xi)}
    }
}

%% Macro for the MC bloc 
% 1: style
% 2: id 
% 3: loc
\newcommand{\MCBloc}[3][]{
    \MCBlocRAW[#1]{#2}{#3};
    \node at (#2/center) {\fontS $\dpStateModMC$};
}

%% Macro for the MCinverse bloc 
% 1: style
% 2: id 
% 3: loc
\newcommand{\MCBlocInv}[3][]{
    \MCBlocRAW[#1]{#2}{#3};
    \node at (#2/center) {\fontS $\dpStateModMCInv$};
}

%%%%% MAIN DRAWING %%%%%%%%%%%%%
\coordinate (D00) at (0,0);

% Draw all the DFFMUX instances
\foreach \xi in {0,...,3} {
    \foreach \yi in {0,...,3} {
        \pgfmathsetmacro\xshDFF{\spacexDFF*\xi}
        \pgfmathsetmacro\yshDFF{\spaceyDFF*\yi}
        \pgfmathsetmacro\DFFindex{int(12-4*\xi+\yi)}
        \DFFMUX{D\DFFindex}{($(D00)+(\xshDFF,-\yshDFF)$)}
        % Compute byte index
        \node at (D\DFFindex/text) {\Large $\DFFindex$};
        % Add the text to the DFF input
        \pgfmathsetmacro\mBound{int(8*\DFFindex)}
        \pgfmathsetmacro\MBound{int(8*(1+\DFFindex))}
        % Add the control of the mux
        \muxCtrl{D\DFFindex/mux}{$\dpStateCtrlRouteIn$}{1}{0}
        % IO plaintext
        \node [anchor=east] (IL) at 
        ($(D\DFFindex/in1)+(-0.5,1.0*\heightDFF)$){\fontS $\AESdpStatePlaintext[\MBound d-1 : \mBound d]$};
        \draw [->,line width=\lwWire] (IL.east) -- ++(0.2,0)  |- (D\DFFindex/in1);
    }
}

%% Compute reference point where to have the x-coordinate of the
% singal to sbox
\coordinate (offsetToSB) at (1,\spaceyXorOut);
\coordinate (offsetToMCInv) at (1,\spaceyXorOutMC);

% Draw KEYXOR value unit
\foreach \xi in {0,5,10,15}{
    \KEYXOR{KX\xi}{($(D\xi/out)+(\spaceXOR,0)$)}
    % IO ports for KEYXOR
    \pgfmathsetmacro\idxV{int(\xi/5)}
    \pgfmathsetmacro\mB{int(8*\idxV)}
    \pgfmathsetmacro\MB{int(8*(\idxV+1))}
    \node [anchor=east] (Label) at (KX\xi/in2) {\fontS $\AESdpStateFromKey [\MB d-1:\mB d]$};
    \node [anchor=east] at (KX\xi/in1) {\fontS $0$};  
    %%% IO port out for KEYXOR
    \pgfmathsetmacro\IdxColRight{int(\xi/5)}
    \node [anchor=west] (Label_out) at ($(D\IdxColRight/out)+(offsetToSB)$) {\fontS $\dpStateToSBForward [\MB d -1:\mB d]$}; 
    \draw [->, line width=\lwWire] (KX\xi/out1) |- (Label_out.west);
}

% Control signal for the KEYXOR forward
\muxCtrl{KX0/mux}{$\dpStateCtrlEnableKAddDual$}{0}{1}
\foreach \xi in {5,10,15}{
    %%% Control signal of the mux
    \muxCtrl{KX\xi/mux}{$\dpStateCtrlEnableKAdd$}{0}{1}
    
}

% Draw KEYXOR value unit for reverse
\node [anchor=west] (Label_out) at ($(D0/out)+(offsetToMCInv)$) {\fontS $\dpStateToMCInverse [8d-1 : 0d]$}; 
\draw [->, line width=\lwWire] (KX0/out1) |- (Label_out.west);
\foreach \xi in {1,2,3}{
    \KEYXOR{KXinv\xi}{($(D\xi/out)+(\spaceXOR,0)$)}
    % IO ports for KEYXOR
    \pgfmathsetmacro\idxV{int(\xi-1)}
    \pgfmathsetmacro\mB{int(8*\idxV)}
    \pgfmathsetmacro\MB{int(8*(\idxV+1))}
    \node [anchor=east] (Label) at (KXinv\xi/in2) {\fontS $\AESdpStateFromKeyInverse [\MB d-1 : \mB d]$};
    \node [anchor=east] at (KXinv\xi/in1) {\fontS $0$};  
    %%% Control signal of the mux
    \muxCtrl{KXinv\xi/mux}{$\dpStateCtrlEnableKAddInverse$}{0}{1}
    %%% IO port out for KEYXOR
    \pgfmathsetmacro\IdxColRight{int(\xi)}
    \pgfmathsetmacro\mB{int(8*\IdxColRight)}
    \pgfmathsetmacro\MB{int(8*(\IdxColRight+1))}
    \node [anchor=west] (Label_out) at ($(D\IdxColRight/out)+(offsetToMCInv)$) {\fontS $\dpStateToMCInverse [\MB d-1:\mB d]$}; 
    \draw [->, line width=\lwWire] (KXinv\xi/out1) |- (Label_out.west);
}


% Draw the MUXIN structure
\foreach \xi in {12,...,15}{
    \MUXIN{MI\xi}{($(D\xi/in2)+(-\spacexIn,0)$)}
    \draw [line width=\lwWire] (MI\xi/out) -- (D\xi/in2);
    % Add the ctrl signal of the muxes
    \muxCtrl{MI\xi/muxFB}{$\dpStateCtrlRouteLoop$}{0}{1}
    \muxCtrl{MI\xi/muxIn}{$\dpStateCtrlRouteMC$}{0}{1}
}

% Draw the mux from SB used in reverse
\node[line width=\scaleCTIKZ*\lwModule,mux2,anchor=rpin 1,xshift=-\xshiftMuxFromSBInv] (muxFSBInv5) at (D1/in2) {};
\muxCtrl{muxFSBInv5}{$\dpStateCtrlEnableToSBInverse$}{0}{1}
\node[anchor=east, yshift=1cm, xshift=-1cm] (FSB5) at (KX5/mux.lpin 1) {\fontS $\AESdpStateFromSB[16d-1:8d]$};
\draw [line width=\lwWire, <-] (muxFSBInv5.lpin 2) -- ++(-1.5,0) |- (FSB5.east);

\node[line width=\scaleCTIKZ*\lwModule,mux2,anchor=rpin 1,xshift=-\xshiftMuxFromSBInv] (muxFSBInv10) at (D6/in2) {};
\muxCtrl{muxFSBInv10}{$\dpStateCtrlEnableToSBInverse$}{0}{1}
\node[anchor=east, yshift=1cm, xshift=-1cm] (FSB10) at (KX10/mux.lpin 1) {\fontS $\AESdpStateFromSB[24d-1:16d]$};
\draw [line width=\lwWire, <-] (muxFSBInv10.lpin 2) -- ++(-1.5,0) |- (FSB10.east);

\node[line width=\scaleCTIKZ*\lwModule,mux2,anchor=rpin 1,xshift=-\xshiftMuxFromSBInv] (muxFSBInv15) at (D11/in2) {};
\muxCtrl{muxFSBInv15}{$\dpStateCtrlEnableToSBInverse$}{0}{1}
\node[anchor=east, yshift=1cm, xshift=-1cm] (FSB15) at (KX15/mux.lpin 1) {\fontS $\AESdpStateFromSB[32d-1:24d]$};
\draw [line width=\lwWire, <-] (muxFSBInv15.lpin 2) -- ++(-1.5,0) |- (FSB15.east);

% Draw the path between registers
\draw [->, line width=\lwWire] (D12/out) |- (D8/in2);
\draw [->, line width=\lwWire] (D8/out) |- (D4/in2);
\draw [->, line width=\lwWire] (D4/out) |- (D0/in2);
\draw [->, line width=\lwWire] (KX0/out2) -| ++(1,-\spaceyFeedBack) -| ($(MI12/inFB)+(-\spacexFeedBack,0)$) -- (MI12/inFB);

\draw [->, line width=\lwWire] (D13/out) |- (D9/in2);
\draw [->, line width=\lwWire] (D9/out) |- (D5/in2);
\draw [->, line width=\lwWire] (KX5/out2) -- (muxFSBInv5.lpin 1);
\draw [->, line width=\lwWire] (muxFSBInv5.rpin 1) -- (D1/in2);
\draw [->, line width=\lwWire] (D1/out) -| ++(1,-\spaceyFeedBack) -| ($(MI13/inFB)+(-\spacexFeedBack,0)$) -- (MI13/inFB);

\draw [->, line width=\lwWire] (D14/out) |- (D10/in2);
\draw [->, line width=\lwWire] (KX10/out2) -- (muxFSBInv10.lpin 1);%(D6/in2);
\draw [->, line width=\lwWire] (muxFSBInv10.rpin 1) -- (D6/in2);
\draw [->, line width=\lwWire] (D6/out) |- (D2/in2);
\draw [->, line width=\lwWire] (D2/out) -| ++(1,-\spaceyFeedBack) -| ($(MI14/inFB)+(-\spacexFeedBack,0)$) -- (MI14/inFB);

\draw [->, line width=\lwWire] (KX15/out2) -- (muxFSBInv15.lpin 1);%(D11/in2);
\draw [->, line width=\lwWire] (muxFSBInv15.rpin 1) -- (D11/in2);
\draw [->, line width=\lwWire] (D11/out) |- (D7/in2);
\draw [->, line width=\lwWire] (D7/out) |- (D3/in2);
\draw [->, line width=\lwWire] (D3/out) -| ++(1,-\spaceyFeedBack) -| ($(MI15/inFB)+(-\spacexFeedBack,0)$) -- (MI15/inFB);

% Draw the MC logic bloc and the input of the last column
\MCBloc{MC}{($(D8/text)!0.5!(D4/text)+(0,16)$)}
\foreach \xi in {0,...,3} {
    %% MC value
    \pgfmathsetmacro\mB{int(8*\xi)}
    \pgfmathsetmacro\MB{int(8*(1+\xi))}
    \pgfmathsetmacro\MCidx{int(\xi+1)}
    \node [anchor=east,xshift=-1.5cm] (toMC) at (MC/a\MCidx) {\fontS $\AESdpStateFromSB[\MB d-1:\mB d]$};
    \draw [->,line width=\lwWire] (toMC.east) -- (MC/a\MCidx);
    \node [anchor=west,xshift=1.5cm] (fromMC) at (MC/b\MCidx) {\fontS $\texttt{fromMC}[\MB d-1:\mB d]$};
    \draw [->,line width=\lwWire] (MC/b\MCidx) -- (fromMC.west);
    %% to last column value
    \pgfmathsetmacro\SmB{int(8*(12-\xi))}
    \pgfmathsetmacro\SMB{int(8*(13-\xi))}
    \pgfmathsetmacro\MUXidx{int(12+\xi)}
    \node [anchor=east] (fSBCmux) at (MI\MUXidx/inSB) {\fontS $\AESdpStateFromSB[\MB d-1: \mB d]$};
    \node [anchor=east] (fMCmux) at (MI\MUXidx/inMC) {\fontS $\texttt{fromMC}[\MB d-1:\mB d]$};
}

% Draw the MC inverse logic bloc 
\MCBlocInv{MCInv}{($(D8/text)+(0,8)$)}
\coordinate (MCIin) at ($(MCInv/a2)!0.5!(MCInv/a3) + (-1.5,0)$);
\coordinate (MCIout) at ($(MCInv/b2)!0.5!(MCInv/b3) + (1.5,0)$);
\debugN[MCIin]{(MCIin)}
\debugN[MCIout]{(MCIout)}
\node [anchor=south east, xshift=-1cm] (MCInvInput) at ($(MCIin)$) {\fontS $\dpStateToMCInverse$};
\node[line width=\scaleCTIKZ*\lwModule,mux2,anchor=lpin 2,xshift=1cm] (muxInvMC) at ($(MCIout)$) {};

% Connexion to MCInv port
\foreach \xi in {1,...,4}{
    \draw [dashed, line width=\lwWire] (MCIin) -- (MCInv/a\xi);
    \draw [dashed, line width=\lwWire] (MCInv/b\xi) -- (MCIout);
}
\draw [line width=\lwWire] (MCInvInput.south west) -- (MCIin);
\draw [->, line width=\lwWire] (MCIout) -- (muxInvMC.lpin 2);
\draw [->, line width=\lwWire] ($(MCInvInput.south east)!0.5!(MCIin)$) -- ++(0,4) -| ($(muxInvMC.lpin 1)+(-0.5,0)$) -- (muxInvMC.lpin 1);

% Mux selecting toSB forward or reverse 
\node[line width=\scaleCTIKZ*\lwModule,mux2,anchor=lpin 1,xshift=7cm] (muxInvToSB) at (muxInvMC.rpin 1) {};
\draw [->, line width=\lwWire] (muxInvMC.rpin 1) -- (muxInvToSB.lpin 1);
\node[anchor=south east, xshift=-0.3cm] (toSBinverse) at (muxInvToSB.lpin 1) {\fontS $\dpStateToSBReverse$};
\node[anchor=south east, xshift=-0.3cm] (toSBforward) at (muxInvToSB.lpin 2) {\fontS $\dpStateToSBForward$};
\draw [->, line width=\lwWire] (toSBforward.south west) -- (muxInvToSB.lpin 2);
\node[anchor=south west, xshift=0.3cm] (toSBpin) at (muxInvToSB.rpin 1) {\fontS $\AESdpStateToSB$};
\draw [->, line width=\lwWire] (muxInvToSB.rpin 1) -- (toSBpin.south east);

\muxCtrl{muxInvMC}{$\dpStateCtrlBypassMCInverse$}{1}{0}
\muxCtrl{muxInvToSB}{$\dpStateCtrlEnableToSBInverse$}{1}{0}

%



