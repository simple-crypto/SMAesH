% define a mux
\tikzset{mux/.style={muxdemux,muxdemux def={Lh=4, NL=2, Rh=3,NB=1,w=1}}}

%%%% Config 
\def\blocLw{0.5mm}
\def\ctLw{0.3mm}
\def\wireLw{0.4mm}
\def\borderLw{0.7mm}

\def\fsizeTop{\Large}
\def\fontS{\Large}
\def\fontCtrl{\Large}

\def\spacexMuxGate{3cm}
\def\xshMuxToSB{4cm}
\def\xshSboxOut{1cm}

\def\sizeB{0.9mm}

% Dot use for Ports
\def\Rad{5pt}
\tikzset{
dot/.style = {circle, fill, minimum size=#1,
              inner sep=0pt, outer sep=0pt},
dot/.default = 6pt % size of the circle diameter 
}

% W internal port macro with annotation
% #1 style
% #2 id
% #3 loc
% #4 text
\def\yshw{3mm}
\def\xshw{2mm}
\newcommand{\portW}[4][]{
    \node [dot=\Rad] (ncirc) at (#3) {};
    \node [#1,anchor=west] at (#3) {#4};
    \coordinate (#2) at (ncirc.west);
    \coordinate (#2/m) at (ncirc.east);
}
% E internal port macro with annotation
% #1 style
% #2 id
% #3 loc
% #4 text
\newcommand{\portE}[4][]{
    \node [dot=\Rad] (ncirc) at (#3) {};
    \node [#1,anchor=east] at (#3) {#4};
    \coordinate (#2) at (ncirc.east);
    \coordinate (#2/m) at (ncirc.west);
}
% S internal port macro with annotation
% #1 style
% #2 id
% #3 loc
% #4 text
\newcommand{\portS}[4][]{
    \node [dot=\Rad] (ncirc) at (#3) {};
    \node [#1,anchor=east,rotate=270] at (#3) {#4};
    \coordinate (#2) at (ncirc.south);
    \coordinate (#2/m) at (ncirc.north);
}
% N internal port macro with annotation
% #1 style
% #2 id
% #3 loc
% #4 text
\newcommand{\portN}[4][]{
    \node [dot=\Rad] (ncirc) at (#3) {};
    \node [#1,anchor=west,rotate=270] at (#3) {#4};
    \coordinate (#2) at (ncirc.north);
    \coordinate (#2/m) at (ncirc.south);
}

%%%%%% TOP arrow
\def\awidth{0.4mm}
\def\xarr{2cm}
% W arrow port macro
% #1 port connected
\newcommand{\arrW}[2][]{
    \draw [#1,line width = \awidth] ($(#2)+(-\xarr,0)$) -- (#2);
}
% E arrow port macro
% #1 port connected
\newcommand{\arrE}[2][]{
    \draw [#1,line width = \awidth] (#2) -- ++(\xarr,0);
}
% S arrow port macro
% #1 port connected
\newcommand{\arrS}[2][]{
    \draw [#1,line width = \awidth] (#2) -- ++(0,-\xarr);
}

%% Macro for the ctrl signals of the mux2
% 1: mux_id
% 2: control sig
% 3: top value
% 4: bottom value
\newcommand{\muxCtrl}[4]{
    \node[anchor=west,rotate=270] at (#1.bpin 1) {\fontCtrl #2};
    \node at (#1.center up) {\fontCtrl #3};
    \node at (#1.center down) {\fontCtrl #4};
}

%% Sbox layer
% #1: draw param
% #2: id
% #3: center loc
\newcommand{\sboxLayer}[3][]{
    % Config
    \def\sbWidth{9}
    \def\sbHeight{2}
    % Draw the Basic rectangle 
    \rectangle[#1]{rec}{#3}{\sbWidth}{\sbHeight}{1}{1}{1}{0}
    \draw node[rectangle,draw,line width=0.6mm,anchor=north west] at (rec/TL) {\fontS $4 \times \modAESsbox$};
    % Draw the pipelie triangles
    \pgfmathsetmacro\trigBase{\sbHeight/3}
    \foreach \xi in {0,...,3}{
        \coordinate (trigCB) at ($(rec/BL) + (0.5*\sbWidth+-2.5*\trigBase,0) + (\xi*\trigBase,0)$);        
        \coordinate (trigC0) at ($(trigCB) + (-\trigBase/2,0)$);
        \coordinate (trigC1) at ($(trigCB) + (0,\trigBase)$);
        \coordinate (trigC2) at ($(trigCB) + (\trigBase/2,0)$);
        \draw [#1] (trigC0) -- (trigC1) --(trigC2);
    }
    % Create ports
    \portW{#2/in}{rec/W1}{\fontS \AESsboxIn}
    \portE{#2/out}{rec/E1}{\fontS \AESsboxOut}
    \portN{#2/rnd}{rec/N1}{\fontS \AESsboxRnd}
}

%% Datapath module for the state
% #1: draw param
% #2: id
% #3: center loc
\newcommand{\dpState}[3][]{
    % Draw rect
    \rectangle[#1]{rec}{#3}{9}{3}{2}{2}{6}{0}
    \draw node[rectangle,draw,line width=0.6mm,anchor=north west] at (rec/TL) {\fontS \modAESdpState};
    % Draw ports
    \portW{#2/plaintext}{rec/W1}{\fontS \AESdpStatePlaintext}
    \portW{#2/from_SB}{rec/W2}{\fontS \AESdpStateFromSB}
    \portN{#2/from_key}{rec/N3}{}%\AESdpStateFromKey
    \portN{#2/from_key_inverse}{rec/N4}{}%\AESdpStateFromKey
    \portE{#2/to_SB}{rec/E2}{\fontS \AESdpStateToSB}
    \portE{#2/ciphertext}{rec/E1}{\fontS \AESdpStateCiphertext}
    % Generating coordinate
    \coordinate (#2/BL) at (rec/BL);
    \coordinate (#2/north) at (rec/north);
}

%% Datapath module for the key
% #1: draw param
% #2: id
% #3: center loc
\newcommand{\dpKey}[3][]{
    % Draw rect
    \rectangle[#1]{rec}{#3}{9}{10}{5}{2}{0}{6}
    \draw node[rectangle,draw,line width=0.6mm,anchor=north west] at (rec/TL) {\fontS \modAESdpKey};
    % Draw ports
    \portW{#2/key}{rec/W2}{\fontS \AESdpKeyKey}
    \portE{#2/to_SB}{rec/E1}{\fontS \AESdpKeyToSB}
    \portW{#2/from_SB}{rec/W1}{\fontS \AESdpKeyFromSB}
    \portS{#2/to_AK}{rec/S3}{\fontS \AESdpKeyToAK}
    \portS{#2/to_AK_inv}{rec/S4}{\fontS \AESdpKeyToAKInverse}
    % Draw generating coordinate
    \coordinate (#2/TL) at (rec/TL);
    \coordinate (#2/south) at (rec/south);
}

%%%% Draw main drawing
\coordinate (locdpState) at (0,0);

%%% Draw dpState + muxes to SB + input muxes
\dpState[line width=\blocLw]{dpState}{(locdpState)}
\node[mux,line width=\ctLw,anchor=lpin 2, xshift=\xshMuxToSB] (mux_to_SB) at (dpState/to_SB) {};
\node[mux,line width=\ctLw,anchor=lpin 2, xshift=1cm] (mux_gate_SB) at (mux_to_SB.rpin 1) {};
\node[mux,line width=\ctLw, anchor=rpin 1,yshift=7cm] (mux_gate_cipher) at (mux_gate_SB.rpin 1) {};
\node[mux,line width=\ctLw,anchor=rpin 1,xshift=-\spacexMuxGate] (mux_gate_plain) at (dpState/plaintext) {};

% Ctrl signals
\muxCtrl{mux_to_SB}{\AESsboxFeedKey}{1}{0}
\muxCtrl{mux_gate_SB}{\AESsboxValidIn}{0}{1}
\muxCtrl{mux_gate_cipher}{\portAESOutValid}{0}{1}
\muxCtrl{mux_gate_plain}{\AESFetchIn}{0}{1}

%% Draw Sbox
\sboxLayer[line width=\blocLw]{sbox}{($(mux_gate_SB.rpin 1) + (6,0)$)}

%% Draw dpKey + input mux
\dpKey[line width=\blocLw]{dpKey}{($(dpState/north) + (0,6)$)}
\node[mux,line width=\ctLw,anchor=rpin 1,xshift=-\spacexMuxGate] (mux_gate_key) at (dpKey/key) {};
\muxCtrl{mux_gate_key}{\AESFetchIn}{0}{1}

%% Draw internal wires
\draw [->,line width=\sizeB] (mux_gate_plain.rpin 1) -- ++(1,0) |- (dpState/plaintext);
\draw [->,line width=\sizeB] (mux_gate_key.rpin 1) -- ++(1,0) |- (dpKey/key);
\draw [->,line width=\wireLw] (dpState/to_SB) -- (mux_to_SB.lpin 2);
\draw [->,line width=\wireLw] (mux_to_SB.rpin 1) -- (mux_gate_SB.lpin 2);
\draw [->,line width=\wireLw] (mux_gate_SB.rpin 1) -- (sbox/in);
\draw [->,line width=\wireLw] (mux_gate_SB.rpin 1) -- (sbox/in);
\draw [->,line width=\wireLw] (dpKey/to_SB) -- ++(0.66*\xshMuxToSB,0) |- (mux_to_SB.lpin 1);
\draw [->,line width=\sizeB] (dpState/ciphertext) -- ++(0.33*\xshMuxToSB,0) |- (mux_gate_cipher.lpin 2);
\draw [->,line width=\wireLw] (dpKey/to_AK) -- (dpState/from_key);
\draw [->,line width=\wireLw] (dpKey/to_AK_inv) -- (dpState/from_key_inverse);

%% Draw feedback wires
\coordinate (sboxOut) at ($(sbox/out)+(\xshSboxOut,0)$);
\debugN[sboxOut]{(sboxOut)}

\coordinate (fbdpKey) at ($(dpKey/TL)+(-1,1)$);
\debugN[fbdpKey]{(fbdpKey)}

\coordinate (fbdpState) at ($(dpState/BL)+(-1,-4)$);
\debugN[fbdpKey]{(fbdpState)}

\draw [line width=\wireLw] (sbox/out) -- (sboxOut);
\draw [->, line width=\wireLw] (sboxOut) |- (fbdpKey) |- (dpKey/from_SB);
\draw [->, line width=\wireLw] (sboxOut) |- (fbdpState) |- (dpState/from_SB);

%% Draw the 0
\node [xshift=-0.2cm] at (mux_gate_cipher.lpin 1) {\fontS $0$};
\node [xshift=-0.2cm] at (mux_gate_plain.lpin 1) {\fontS $0$};
\node [xshift=-0.2cm] at (mux_gate_key.lpin 1) {\fontS $0$};
\node [xshift=-0.2cm] at (mux_gate_SB.lpin 1) {\fontS $0$};


%% Draw port
\draw node[rectangle,anchor=east,xshift=-1cm] (top_plaintext) at (mux_gate_plain.lpin 2){\fsizeTop \portAESInPlaintext [128d-1:0]};
\draw node[rectangle,anchor=east,xshift=-1cm] (top_key) at (mux_gate_key.lpin 2){\fsizeTop \portAESInKey [256d-1:0]};
\draw node[rectangle,anchor=west,xshift=1cm] (top_ciphertext) at (mux_gate_cipher.rpin 1){\fsizeTop \portAESOutCipher [128d-1:0]};
\draw node[rectangle,anchor=east,rotate=270,xshift=-1cm] (top_rnd) at (sbox/rnd){\fsizeTop \portAESRnd };

\draw [->,line width = \sizeB] (top_plaintext.east) -- (mux_gate_plain.lpin 2);
\draw [->,line width = \sizeB] (top_key.east) -- (mux_gate_key.lpin 2);
\draw [->,line width = \sizeB] (mux_gate_cipher.rpin 1) -- (top_ciphertext.west);
\draw [->,line width = \sizeB] (top_rnd.east) -- (sbox/rnd);

